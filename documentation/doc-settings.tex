\usepackage{footnote} 
\makesavenoteenv{tabular} % Makes it possible to use footnotes in tabular environments
\definecolor{documentation}{HTML}{974228}
\usepackage{listings}
\lstdefinestyle{bash}{
  language=bash,
  basicstyle=\ttfamily,
  keywordstyle=\color{documentation},
  alsoletter={-.0123456789},
  morekeywords={
    lualatex,
    biber,
    lualatex,
    lualatex
  },
}

\lstdefinestyle{latex}{
  language=[LaTeX]Tex,
  basicstyle=\ttfamily,
  texcsstyle=\color{documentation},
  moredelim =*[s][\color{gray}]{<}{>},
  moredelim =**[is][\color{documentation!50}]{//ann\{}{\}},
  moretexcs={
    maketitle,
    frontmatter,
    makefacultytitle,
    Preface,
    tableofcontents,
    listoffigures,
    listoftables,
    mainmatter,
    Introduction,
    chapter,
    subsection,
    subsubsection,
    appendix,
    faupressprintbibliography,
    faupressprintacronyms,
    DeclareAcronym,
    ac,
    acs,
    si,
    meter,
    per,
    second,
    includegraphics,
    graphicspath,
    toprule,
    midrule,
    bottomrule,
    autoref,
    nameref,
    SI,
    cm,
    micro,
    metre,
    acl,
    sisetup,
    firstname,
    lastname,
    degree,
    origin,
    subtitle,
    institute,
    supervisor,
    series,
    volume,
    doi,
    isbn,
    eisbn,
    issn,
    printinformation,
    oralexam,
    dean,
    reviewer
  },
}

\usepackage{tcolorbox}
\tcbuselibrary{listings,skins,xparse}

\tcbset{
  custom/.style={
    enhanced,
    overlay={
      \node at ([xshift = -1.5cm,yshift=-0.3cm]frame.north east)
      {\itshape\footnotesize\color{documentation}Custom command};
    }
  },
  acronym/.style={
    enhanced,
    enlarge top by = 0.5mm,
    overlay={
      \node at ([xshift = -1.2cm,yshift=0.2cm]frame.north east)
      {\ttfamily\footnotesize\color{documentation}acronym.tex};
    }
  },
  bib/.style={
    enhanced,
    enlarge top by = 0.5mm,
    overlay={
      \node at ([xshift = -1.55cm,yshift=0.2cm]frame.north east)
      {\ttfamily\footnotesize\color{documentation}bibliography.bib};
    }
  },
  colback = white,
  colframe = documentation,
}

\NewTCBListing{codebox}{ !O{} }{%
  colback = white,
  colframe = documentation,
  listing only,
  listing options = {style=latex},
  #1
}

\NewTCBListing{codeexamplebox}{ !O{} }{%
  colback = white,
  colframe = documentation,
  listing options = {style=latex},
  #1
}
\lstset{style=latex}
\newcommand{\code}{\lstinline}
\newcommand{\secref}[1]{\textit{\nameref{#1}} on page~\pageref{#1}} % Command to reference
% sections in style
\newcommand{\urlfootnote}[1]{\footnote{\url{#1}}}
\newcommand{\important}[1]{\textbf{#1}}