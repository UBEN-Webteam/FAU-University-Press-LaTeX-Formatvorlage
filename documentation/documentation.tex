\documentclass[
  paper = 17x24, % studienpartitur, a5, a5paper
  language = english, % german
  acronym = true, % none, true, false, combined, onlyabbreviation, onlysymbol, noabbreviation, nosymbol
  bibliography = true, % none, split, false
  acronymline = onlyhorizontal, % default, true, false, both, all, none, nohorizontal,
                                % onlyhorizontal, novertical, onlyvertical, 
]{faupress}

\firstname{FAU}
\lastname{University Press}
\yearofpublication{2020}
\degree{M.Sc.}
\origin{Erlangen}
\title{Documentation}
\subtitle{For the LaTeX template}
\institute{University Library of Erlangen-Nürnberg}
\supervisor{Supervisor}
\series{FAU Studien aus dem Maschinenbau}
\volume{XXX}
\doi{10.25593/978-3-96147-XXX-X}
\isbn{978-3-96147-XXX-X}
\eisbn{978-3-96147-XXX-X}
\issn{2625-9974}
\printinformation{ggf. Satz \\%
  ggf. Druck}
\oralexam{12.12.1769}
\dean{Daniel de Superville}
\reviewer{Friedrich III. von Brandenburg-Bayreuth \\
  Christian Friedrich Carl Alexander von Brandenburg-Ansbach}

% Only included for the documentation
\usepackage{footnote} 
\makesavenoteenv{tabular} % Makes it possible to use footnotes in tabular environments
\definecolor{documentation}{HTML}{974228}
\usepackage{listings}
\lstdefinestyle{bash}{
  language=bash,
  basicstyle=\ttfamily,
  keywordstyle=\color{documentation},
  alsoletter={-.0123456789},
  morekeywords={
    lualatex,
    biber,
    lualatex,
    lualatex
  },
}

\lstdefinestyle{latex}{
  language=[LaTeX]Tex,
  basicstyle=\ttfamily,
  texcsstyle=\color{documentation},
  moredelim =*[s][\color{gray}]{<}{>},
  moredelim =**[is][\color{documentation!50}]{//ann\{}{\}},
  moretexcs={
    maketitle,
    frontmatter,
    makefacultytitle,
    Preface,
    tableofcontents,
    listoffigures,
    listoftables,
    mainmatter,
    Introduction,
    chapter,
    subsection,
    subsubsection,
    appendix,
    faupressprintbibliography,
    faupressprintacronyms,
    DeclareAcronym,
    ac,
    acs,
    si,
    meter,
    per,
    second,
    includegraphics,
    graphicspath,
    toprule,
    midrule,
    bottomrule,
    autoref,
    nameref,
    SI,
    cm,
    micro,
    metre,
    acl,
    sisetup,
    firstname,
    lastname,
    degree,
    origin,
    subtitle,
    institute,
    supervisor,
    series,
    volume,
    doi,
    isbn,
    eisbn,
    issn,
    printinformation,
    oralexam,
    dean,
    reviewer
  },
}

\usepackage{tcolorbox}
\tcbuselibrary{listings,skins,xparse}

\tcbset{
  custom/.style={
    enhanced,
    overlay={
      \node at ([xshift = -1.5cm,yshift=-0.3cm]frame.north east)
      {\itshape\footnotesize\color{documentation}Custom command};
    }
  },
  acronym/.style={
    enhanced,
    enlarge top by = 0.5mm,
    overlay={
      \node at ([xshift = -1.2cm,yshift=0.2cm]frame.north east)
      {\ttfamily\footnotesize\color{documentation}acronym.tex};
    }
  },
  bib/.style={
    enhanced,
    enlarge top by = 0.5mm,
    overlay={
      \node at ([xshift = -1.55cm,yshift=0.2cm]frame.north east)
      {\ttfamily\footnotesize\color{documentation}bibliography.bib};
    }
  },
  colback = white,
  colframe = documentation,
}

\NewTCBListing{codebox}{ !O{} }{%
  colback = white,
  colframe = documentation,
  listing only,
  listing options = {style=latex},
  #1
}

\NewTCBListing{codeexamplebox}{ !O{} }{%
  colback = white,
  colframe = documentation,
  listing options = {style=latex},
  #1
}
\lstset{style=latex}
\newcommand{\code}{\lstinline}
\newcommand{\secref}[1]{\textit{\nameref{#1}} on page~\pageref{#1}} % Command to reference
% sections in style
\newcommand{\urlfootnote}[1]{\footnote{\url{#1}}}
\newcommand{\important}[1]{\textbf{#1}}

\begin{document}
\maketitle % Typesets the titlepage of the publisher

\frontmatter % Starts the frontmatter (page numbering, roman numerals, no header)
\makefacultytitle % Typesets the titlepage of the faculty

\begin{preface}

This is the documentation of the \LaTeX{} template for publications in the series
\enquote{FAU Studien aus dem Maschinenbau}. This guide provides an overview of the changes
made to the standard \code+scrbook+ class (on which this class is based on) that are
important to the end user.

It also introduces some packages that are automatically loaded by the class. For the
packages for acronyms and bibliographies class specific commands have been
defined to meet the typographical requirements. A list of all packages loaded by the class
can be found under \secref{cha:packages}.

This documentation covers only a very small part of the features of \LaTeX{} and the included
packages. It is always worth taking a look at the documentation of the individual packages.

A detailed introduction to \LaTeX{} in general can be found at \linebreak
\url{https://www.overleaf.com/learn/latex/Main_Page}.

\end{preface}

\tableofcontents\label{toc}
\faupressprintacronyms
\listoffigures\label{lof}
\listoftables\label{lot}

\mainmatter % Arabic numerals

\begin{introduction}

The class \code+faupress+ implements all format specifications of the \acs{fau} University
Press. To use this class in its entirety, it \important{should be compiled with
  Lua\LaTeX}.

To achieve better consistency, some new commands have been defined and some old ones have
been redefined. Such commands were marked with \textcolor{documentation}{\itshape Custom command} in the following documentation.

\end{introduction}

\chapter{Class options}

The class is loaded by the following command:
\begin{codebox}
  \documentclass[<options>]{faupress}
\end{codebox}
There are several options available that can be passed as key-value pairs. Multiple
options are separated by commas. Below you can see the default options and their alternatives.

The \code+paper+ option can be used to change the paper size:
\begin{codebox}
  paper = 17x24 //ann{default}
          a5
\end{codebox}
There are two languages available:
\begin{codebox}
  language = german //ann{default}
             english
\end{codebox}
Setting the language takes care of the microtypography, changes language specific
commands (e.g. \code+\tableofcontents+) and  switches to the correct hyphenation
pattern. The \code+polyglossia+ package is used for this. The language that was not
selected is still set as a second language. This allows individual sections to be set in
the other language. The package provides an environment for this, e.g. for German:
\begin{codeexamplebox}
  Today is \today.
  \begin{german}
    Heute ist der \today.
  \end{german}
\end{codeexamplebox}

The list of abbreviations and symbols is generated using the \code+acro+ package. By
default, separated lists for abbreviations and symbols are printed. If want only one big
list you can use the option \code+combined+. If you only want one of those lists you can
use \code+onlyabbreviation+ and \code+onlysymbol+.
\begin{codebox}
  acronym = true //ann{default}
            combined
            onlyabbreviation
            onlysymnnbol
\end{codebox}
By default the rows of the acronym table are interspersed by horizontal lines
and the columns are separated by vertical lines. This behaviour can be changed
using the following option:
\begin{codebox}
  acronymline = both //ann{default}
                onlyhorizontal
                onlyvertical
                none
\end{codebox}
For more information, see \secref{cha:acronyms}.
          
By default, this class uses the \code+biblatex+ to manage citations. If you do not want to
use this feature you can switch it of:
\begin{codebox}
  bibliography = true //ann{default}
                 none
\end{codebox}
Setting this option to \code+none+ stops \code+biblatex+ and all associated definitions
from being loaded. For more information, see \secref{cha:bibliography}.

The bibliography consists of three parts, the general bibliography, the one of
student's works referring to this work and the one for own publications. If one
wants to disable one of these there is the option:
\begin{codebox}
  bibliographypart = all //ann{default}
                     none
                     nomain
                     onlymain
                     noown
                     onlyown
                     nostudent
                     onlystudent
\end{codebox}

The options for this documentation are:
\begin{codebox}
  \documentclass[
    paper = 17x24,
    language = english,
    acronym = true,
    acronymline = onlyhorizontal,
    bibliography = true
    ]{faupress}
\end{codebox}

\chapter{Structure of the document}

This chapter gives a brief summary of the commands available to structure the
document.

The title pages are created automatically. To display the those its components have to be
declared in the preamble of the document. The storing commands for this class can be found
under \secref{app:storing} in the \nameref{app:appendix}.

\section{Front page}
The front page of this document is generated by the command:
\begin{codebox}[custom]
  \maketitle
\end{codebox}
This typesets the first two pages of the document using the information given in the
preamble.

\section{Front matter}
After the first two pages the front matter is started by the command:
\begin{codebox}
  \frontmatter
\end{codebox}
This command switches on page numbering  using Roman numerals. The first two pages in the
front matter are the title page of the faculty, which are also generated from the storing
commands in the preamble by:
\begin{codebox}[custom]
  \makefacultytitle
\end{codebox}
\newpage
The faculty title page is followed by the preface. The preface chapter is started by
\begin{codebox}[custom]
  \begin{preface}
  \end{preface}
\end{codebox}
This command typesets a unnumbered chapter named \enquote{Preface} (German:
\enquote{Vorwort}). This chapter does not appear in the table of contents, which is
automatically generated from the sectioning commands used in the document by the command:
\begin{codebox}
  \tableofcontents
\end{codebox}
The table of contents for this document can be found on \pageref{toc}. The first chapter
of the front matter to appear in the table of contents is the List of Symbols and
Abbreviations (German: \enquote{Formelzeichen- und Abkürzungsverzeichnis}). With the help
of the \texttt{acro} package the list can be generated automatically:
\begin{codebox}[custom]
  \faupressprintacronyms
\end{codebox}
For more information, see \secref{cha:acronyms}.

The front matter is completed by lists of figures and of tables:
\begin{codebox}
  \listoffigures
  \listoftables
\end{codebox}
Both can be found on pages~\pageref{lof} and \pageref{lot} respectively.

\newpage
\section{Main matter}
The main matter is started by the command:
\begin{codebox}
  \mainmatter
\end{codebox}
Here, the page numbering is reset and displayed with Arabic numerals.

The first chapter in the main part of the document is \enquote{Introduction}
(German: \enquote{Einleitung}) and is typeset by:
\begin{codebox}[custom]
  \begin{introduction}    
  \end{introduction}
\end{codebox}
All chapters \& sections in the main matter are numbered and appear in the table of
contents. This is done using the standard \LaTeX{} commands:
\begin{codebox}
  \chapter{<text>}
  \section{<text>}
  \subsection{<text>}
  \subsubsection{<text>}
\end{codebox}
The body ends with the chapters \enquote{Summary and outlook} and \enquote{Zusammenfassung
  und Ausblick}:
\begin{codebox}
  \chapter{Zusammenfassung und Ausblick}
  \chapter{Summary and outlook}
\end{codebox}

\newpage
\subsection{Short titles}

For long chapter \& section headings it is recommended to use a short title. All
sectioning commands have an optional argument to set a short title.
\begin{codebox}
  \chapter[<short title>]{<long title>}
  \section[<short title>]{<long title>}
  \subsection[<short title>]{<long title>}
  \subsubsection[<short title>]{<long title>}
\end{codebox}
The short title is what appears in the table of contents and in the headline. It should be
used to avoid two-line headlines.

\section{Appendix \& Bibliography}

The chapter \enquote{Appendix} (German: \enquote{Anhang}) is typeset using the command:
\begin{codebox}[custom]
  \appendix
\end{codebox}
This chapter is unnumbered and appears in the table of contents. The
\code+\chapter+ command does not exist in the appendix. The normal sectioning
commands work the same in the appendix. Only sections appear in the table of
contents and are numbered with letters, all other sectioning commands do not
appear in the table of contents and are numbered. It also automatically reduces
the fontsize in the table of contents.
\begin{codebox}
  \section{<text>}
  \subsection{<text>}
  \subsubsection{<text>}
\end{codebox}



The document is completed by the backmatter:
\begin{codebox}
  \backmatter
\end{codebox}
In the backmatter the \code+\chapter+ command is reintroduced and the fontsize
in the table of contents is small than for the main part. It contains the bibliography:
\begin{codebox}[custom]
  \faupressprintbibliography
\end{codebox}
This command creates a general bibliography, a list of own publications, and a list of
students' theses referring to this work. For more information see
\secref{cha:bibliography}.


\chapter{Acronyms}
\label{cha:acronyms}

To typeset acronyms the package \texttt{acro}\urlfootnote{https://ctan.org/pkg/acro} is automatically loaded by the class.

Acronyms are defined in the file \texttt{acronyms.tex} (the class will
automatically load this file). The basic syntax to define an acronym is:
\begin{codebox}
  \DeclareAcronym{<label>}{
    short = <acronym>,
    long  = <text>
  }
\end{codebox}
These acronyms can be called in the body of the document using the command:
\begin{codebox}
 \ac{<label>}
\end{codebox}
At the first appearance, this commands prints the long form of the
acronym followed by the short form in parentheses. After this, each
time \code+\ac{<label>}+ is used only the short form is printed. There
are many variations (capitalization, plural forms, only short form,
only long form) of this command, which can be found in the
official documentation.

To define formula symbols you need to supply extra information. Here, the basic syntax is:
\begin{codebox}
  \DeclareAcronym{<label>}{
    short = \ensuremath{<symbol>},
    sort  = <sort-key>,
    long  = <text>,
    extra = <unit>,
    class = {formula}
  }
\end{codebox}
For math symbols, \code+\ensuremath+ should be used so that the
acronym can be used in text and math environments without problems. As
\code+<sort-key>+ a word must be given. This word is used in the list
of acronyms for sorting. For the formula symbol to appear in the
correct list, it must be assigned to the \texttt{formula} class. 

To display lists of all declared acronyms and symbols use:
\begin{codebox}[custom]
  \faupressprintacronyms[<options>]
\end{codebox}
You can pass the same optional parameters to this command as you would
pass them to \code+\printacronyms+ from the original package.

The \code+\faupressprintacronyms+ command typesets a chapter called
\enquote{List of Symbols and Abbreviations} (German:
\enquote{Formelzeichen- und Abkürzungsverzeichnis}). In this chapter a
list of abbreviations (German: \enquote{Abkürzungsverzeichnis}) and a
list of symbols (German: \enquote{Formelverzeichnis}) are created
separately. Both lists can be combined to one by setting the class
option \code+acronym+ to \code+combined+. If you only want one of those lists you can set
\code+acronym+ to \code+onlyabbreviation+ and \code+onlysymbol+. 

Furthermore, the fixed label \texttt{faupressacronyms} is given to the
chapter for referencing.

The acronym functionality can be switched off completely via the class option
\texttt{noacro}. This option prevents the class from loading the \texttt{acro} 
packages and from defining associated commands.
\newpage
\section{Examples}

We define text acronyms in \texttt{acronyms.tex} as follows:
\begin{codebox}[acronym]
  \DeclareAcronym{fau}{
    short = FAU,
    long  = Friedrich-Alexander-Universät
            Erlangen-Nürnberg
  }
  \DeclareAcronym{latex}{
    short = \LaTeX,
    long  =  Lamport \TeX
  }
\end{codebox}

Now we can use the acronyms in the main text:
\begin{codeexamplebox}
  \ac{fau} was founded in 1742. Only 242 years
  after the foundation of \ac{fau}, \ac{latex}
  was first published.
\end{codeexamplebox}
We already used \code+faupressprintacronyms+ at the beginning of the
document. The list of acronyms can be found on page 
\pageref{faupressacronyms}. On the same page you can only see the list
of symbols. In both lists, all acronyms that were defined are always displayed,
regardless of whether they were used in the main text or not. 

\newpage
Formula symbols are defined as follows:
\begin{codebox}[acronym]
  \DeclareAcronym{c0}{
    short = \ensuremath{c_0},
    sort = {c},
    long = speed of light in vacuum,
    extra = \si{\meter\per\second},
    class = {formula}
  }
  \DeclareAcronym{gamma}{
    short = \ensuremath{\gamma},
    sort = {gamma},
    long = Lorentz factor,
    class = {formula}
  }
\end{codebox}
Here, we typeset the units with help of the \texttt{siunitx} package (see
\secref{sec:units}). Since \code+\ensuremath+ was used, these acronyms can be
used in both text and math mode:
\begin{codeexamplebox}
  The \ac{c0} is needed to define the \ac{gamma}:
  \begin{equation}
    \acs{gamma} = \frac1{\sqrt{1 -
        \left(\frac{v}{\acs{c0}}\right)}^2}
  \end{equation}
\end{codeexamplebox}
It is advisable to use only the short forms of the acronym in math mode. To ensure
this, we use the command \code+\acs+.\footnote{See \texttt{acro}
  documentation: \url{https://ctan.org/pkg/acro}}

\chapter{Figures \& tables}
\label{figures-and-tables}
Figures and tables are placed inside so called floats. These are containers for things in a
document that cannot be broken over a page. They automatically take care of the placement
of graphics and tables in the text and format the caption.
To create a figure that floats, use:
\begin{codebox}
  \begin{figure}[<placement>]
  \end{figure}
\end{codebox}
In an analogous manner, floating tables is created by: 
\begin{codebox}
  \begin{table}[<placement>]
  \end{table}
\end{codebox}
The captions are set inside those floats by:
\begin{codebox}
  \caption{<text>}
\end{codebox}
This command is only available in float environments and should be placed \important{above tables}
and \important{below figures}. It is also possible to refer to floats in the text, for more
on this see \secref{cha:ref}.

\section{Figures}

The \code+graphicx+\urlfootnote{https://ctan.org/pkg/graphicx} package allows the
integration of images into \LaTeX{} documents. This package is automatically loaded by the
class. 

The most important command of this package is:
\begin{codebox}
  \includegraphics[<options>]{<image path>}
\end{codebox}
The \code+<image path>+ can be given as an absolute path or relative to the directory
where the \texttt{.tex} file is located. The options allow you to scale, rotate and crop
images. All options can be found in the documentation.\urlfootnote{https://ctan.org/pkg/graphicx}

Here you can see an example of a full figure:
% codeexamplebox does not work with floats
\begin{tcolorbox}
\begin{lstlisting}
  \begin{figure}
    \centering
    \includegraphics[width=0.5\textwidth]
      {figures/lines.jpg}
    \caption{This is a figure.}
  \end{figure}
\end{lstlisting}
  \tcblower
  \centering
  \includegraphics[width=0.5\textwidth]{lines.jpg}
  \captionof{figure}{This is a figure.}
  \label{example-figure}
\end{tcolorbox}
The image is scaled to half the text width using the \code+width+ option. An entry was
also automatically created in the \secref{lof}.

If there is a dedicated image folder it can be specified with the \code+\graphicspath+
command in the preamble. The default for this class is:
\begin{codebox}
  \graphicspath{
    {./graphics/},
    {./figures/}
  }
\end{codebox}

\section{Tables}
The \code+tabular+ environment can be used to typeset tables with optional horizontal and
vertical lines. \LaTeX{} determines the width of the columns automatically.  
\begin{codeexamplebox}
  \begin{tabular}{| r | c | l |}
    \hline
    This & is a      & table!         \\\hline
    This & table has & three columns. \\
    \hline
  \end{tabular}
\end{codeexamplebox}
The number of columns does not need to be specified as it is inferred by looking at the number of arguments provided. It is also possible to add vertical lines between the columns here. The following symbols are available to describe the table columns:\\[0.5\baselineskip]
\begin{tabular}{r @{\;--\;} l}
  \code+l+ & left-justified column \\
  \code+c+ & centered column \\
  \code+r+ & right-justified column \\
  \code+p{<width>}+ & paragraph column with fixed width \\
  \code+|+ & vertical line \\
  \code+||+ & double vertical line
\end{tabular}\\[0.5\baselineskip]
Three additional column types are defined by this class:\\[0.5\baselineskip]
\begin{tabular}{r @{\;--\;} l}
  \code+L{<width>}+ & left-justified column with fixed width \\
  \code+C{<width>}+ & centered column with fixed width \\
  \code+R{<width>}+ & right-justified column with fixed width \\
\end{tabular}

\newpage
There is a variety of packages available to extend the table functionality of \LaTeX. The
following packages are automatically included by the class:\\[0.5\baselineskip]
\begin{tabular}{r @{\;--\;} p{0.75\textwidth}}
  \code+booktabs+ & Enhances the quality of tables in \LaTeX.\urlfootnote{https://ctan.org/pkg/booktabs}\\
  \code+array+ & An extended implementation of the array and tabular environments which extends the options for column formats, and provides \enquote{programmable} format specifications.\urlfootnote{https://ctan.org/pkg/array} \\
  \code+longtable+ & Allows you to write tables that continue to the next page.\urlfootnote{https://ctan.org/pkg/longtable}\\
\end{tabular}

\newpage
\subsection{Example}
This section shows a somewhat more involved example that uses functionality from the
packages mentioned above.
% codeexamplebox does not work with floats
\begin{tcolorbox}
\begin{lstlisting}
  \begin{table}
    \centering
    \caption{This is a table.}
    \begin{tabular}{>{\bfseries}r l}
      \toprule
             & Last modified \\\midrule
         Day & \the\day   \\
       Month & \the\month \\
        Year & \the\year  \\
      \bottomrule
    \end{tabular}
  \end{table}
\end{lstlisting}
  \tcblower
  \centering
  \captionof{table}{This is a table.}
  \label{example-table}
  \begin{tabular}{>{\bfseries}r l}
    \toprule
    & Last modified \\\midrule
    Day & \the\day      \\
    Month & \the\month    \\
    Year & \the\year     \\
    \bottomrule
  \end{tabular}
\end{tcolorbox}
The \code+array+ package allows you to prefix a column using the \code+>+ operator. This
is used in this example to typeset the first column in bold font. The commands
\code+\toprule+, \code+\midrule+, and \code+\bottomrule+ are provided by the
\code+booktabs+ package. These commands draw horizontal lines with adjusted spacing.

The commands \code+\the\day+, \code+\the\month+, and \code+\the\year+ are standard \TeX{}
commands. They print out the day, month, and year of the day the underlying \texttt{.tex}
document was compiled.

\chapter{Labels and cross-referencing}
\label{cha:ref}
In \LaTeX, we can label entities that are numbered (sections, floats, formulas, etc.), and
then use that label to refer to them elsewhere.

\begin{codebox}
  \label{<marker>}
\end{codebox}
The marker can be seen as a name that we give to the object that we want to
reference. It is important to add \label after a numbered element, otherwise the label
will not \enquote{latch on}.

The number assigned to the object labeled by \code+<marker>+ is printed by the command:
\begin{codebox}
  \ref{<marker>}
\end{codebox}

The page number of the labeled object is printed using:
\begin{codebox}
  \pageref{<marker>}
\end{codebox}

The package \code+hyperref+\urlfootnote{https://ctan.org/pkg/hyperref}, which is loaded by
the class, provides the commands:
\begin{codebox}
  \autoref{<marker>}
  \nameref{<marker>}
\end{codebox}
\code+\autoref+ prints the name of the object the \code+<marker>+ refers to in front of
the number, e.g. \enquote{\autoref{example-figure}}. \code+\nameref+ prints the title of
the object it refers to, e.g.~\enquote{\nameref{cha:ref}} for the current chapter.

Apart from providing these commands, all cross-referenced elements become hyperlinked by
importing the \code+hyperref+ package. In addition, the hyperref package also provides
commands to typeset web links. 

\section{Example}

First, we need to add a label to our example from page \pageref{example-figure}.
The \code+\label+ command has to always be placed below the \code+\caption+ command:
\begin{codebox}
  \begin{figure}
    \centering
    \includegraphics[width=0.5\textwidth]
    {figures/lines.jpg}
    \caption{This is a figure.}
    \label{example-figure}
  \end{figure}  
\end{codebox}
We also add a label to the chapter about figures and tables:
\begin{codebox}
  \chapter{Figures \& tables}
  \label{figures-and-tables}
\end{codebox}
Now, we can use these markers to refer to objects we labeled:
\begin{codeexamplebox}
  \autoref{example-figure} is on
  page~\pageref{example-figure} in
  the chapter \nameref{figures-and-tables}.
\end{codeexamplebox}

\chapter{Math}
This class automatically includes the
\code+mathtools+\urlfootnote{https://www.ctan.org/pkg/mathtools} package and the packages
of the American Mathematical Society
(\code+amsmath+\urlfootnote{https://www.ctan.org/pkg/amsmath},
\code+amsfonts+\urlfootnote{https://www.ctan.org/pkg/amsfonts}, \code+assume+). 

To typeset mathematical expressions as part of the text, the inline mode is used as you
can see in this example:
\begin{codeexamplebox}
  This \(V - E + F = 2\) is an inline equation.
\end{codeexamplebox}
Expressions, which are not part of the text directly, are typeset in display mode. For
numbered equations there is the \code+equation+ environment, which puts the expression on
separate lines:
\begin{codeexamplebox}
  This is a numbered equation
  \begin{equation}
    R_{\mu\nu} - \frac12 g_{\mu\nu} R
      = \frac{8\pi G}{{c_0}^4}T_{\mu\nu}.
  \end{equation}
\end{codeexamplebox}
\newpage
For unnumbered equations in display mode there is a starred version of the
\code+equation*+. The short form for this is \code+\[ \]+:
\begin{codeexamplebox}
  This is unnumbered equation
  \[
    i \hbar \frac\partial{\partial t}
      \Psi = \hat H \Psi.
  \]
\end{codeexamplebox}

This class loads the \code+unicode-math+ packages, which supplies the following
commands for formatting math:
\begin{codeexamplebox}
  \( \symbf{AaBb\nabla\alpha\beta\gamma} \)   \\
  \( \symrm{AaBb\nabla\alpha\beta\gamma} \)   \\
  \( \symfrak{AaBb\nabla\alpha\beta\gamma} \) \\
  \( \symbfit{AaBb\nabla\alpha\beta\gamma}\)  \\
  \dots
\end{codeexamplebox}
Other formatting commands might not work, because of the used fonts.

\section{Units}
\label{sec:units}

The \code+siunitx+\urlfootnote{https://ctan.org/pkg/siunitx} package offers sophisticated
features to typeset units and numbers. Only the basic functions are shown here, but it is
worth taking a look at the documentation. 

The packages offers two commands to typesets units: \code+\si+ and \code+\SI+. The first
one only prints the unit. The second one has two arguments and typeset unit in combination
with a number:
\begin{codeexamplebox}
  \SI{1}{\meter} are \SI{100}{\cm} and lots
  of \si{\micro\meter}.

  The \acl{c0} is
  \SI{299 792 458}{\metre\per\second}.
\end{codeexamplebox}
As a unit you can either use normal text or predefined unit macros, which are available
for all common units.

\newpage
\code+siunitx+ makes it easy to print quantities with uncertainties:
\begin{codeexamplebox}
  This document has \SI{40 +- 5}{pages}.
\end{codeexamplebox}
How the uncertainty is displayed in detail can be controlled via \code+\sisetup+ (There
are innumerable other options). The default for this class is:
\begin{codebox}
  \sisetup{
    separate-uncertainty = true,
    round-mode = places,
    round-precision = 3,
  }
\end{codebox}
As you can see, the package also takes care of rounding units.

\chapter{Bibliography}
\label{cha:bibliography}

\minisec{Important:}
The custom bibliography commands only work with Biber~2.14 (Bib\LaTeX{}~3.14) or
newer. Additionally, due to the implementation of the custom commands, the command
\code+\nocite{}+ cannot be used.


To write and manage bibliographies, this class loads the
\code+biblatex+\urlfootnote{https://ctan.org/pkg/biblatex} package. This package loads
bibliographic information form \texttt{.bib} files and makes it available for citations in
the document. For this class there are three \important{fixed}~\texttt{.bib} files in the
folder \enquote{references}:\\[0.5\baselineskip]
\begin{tabular}{@{}>{\ttfamily}r@{\;--\;} p{0.75\textwidth}}
  bibliography.tex & for general literature you cite in your thesis\\
  bibliography-own.tex & for own publications referring to this work\\
  bibliography-student.tex & for students’ theses referring to this work
\end{tabular}\\[0.5\baselineskip]
A typical entry in such a file looks like this:
\begin{codebox}[bib]
  @book{lamport1994latex,
    title={\LaTeX: a document preparation
      system: user's guide and reference manual},
    author={Lamport, Leslie},
    year={1994},
    publisher={Addison-Wesley},
    edition={2}
  }
\end{codebox}
It begins with the type of source and is followed by a keyword, which is later used for
citing the entry. This keyword is followed by the bibliographic information. All fields
for this can be found in the documentation. While one can edit these \texttt{.bib} files
per hand, it is easier to use reference management software. Almost all of those programs
can export their library to \text{.bib} files.

In the text a source can then be cited using the \code+\cite+ command:
\begin{codeexamplebox}
  The first edition of Leslie Lamport's book
  was published in 1985 \cite{lamport1994latex}.
\end{codeexamplebox}
\newpage
The bibliography is constructed by:
\begin{codebox}[custom]
  \faupressprintbibliography
\end{codebox}
At first this commands prints all sources cited in the document from the file
\texttt{bibliography.bib}. After this, it prints all sources contained in
\texttt{bibliography-own.tex} under the title \enquote{Own publications referring to this
  work} (German: \enquote{Verzeichnis promotionsbezogener, eigener Publikationen}) and all
sources in \texttt{bibliography-student.tex} with the title \enquote{Students’ theses
  referring to this work} (German: \enquote{Verzeichnis promotionsbezogener, studentischer
  Arbeiten}). In the last two cases, all file entries are always printed, regardless of
 whether they were referenced in the document or not.

To generate a bibliography the program \texttt{biber} is used, which performs all sorting,
label generation (and a lot more). In most \LaTeX{} environments you can specify that
\texttt{biber} is executed directly when compiling.

If you compile the file using the command line, the compilation process would look like this:
\begin{codebox}[listing options = {style=bash}]
  lualatex phd_thesis.tex
  biber phd_thesis
  lualatex phd_thesis.tex
  lualatex phd_thesis.tex
\end{codebox}
Lua\LaTeX{} has to be executed twice at the end, so that all references are correct.


\chapter{List of all included packages}
\label{cha:packages}

\textcolor{documentation}{The list is still missing!}

\appendix
\label{app:appendix}
\section*{Storing commands}
\label{app:storing}
The storing commands needed for this class can be seen below. This code is used exactly
like this in the preamble of this document:
\begin{codebox}[custom]
  \firstname{FAU}
  \lastname{University Press}
  \yearofpublication{2020}
  \degree{M.Sc.}
  \origin{Erlangen}
  \title{Documentation}
  \subtitle{For the LaTeX template}
  \institute{University Library of
    Erlangen-Nürnberg}
  \supervisor{Supervisor}
  \series{FAU Studien aus dem Maschinenbau}
  \volume{XXX}
  \doi{10.25593/978-3-96147-XXX-X}
  \isbn{978-3-96147-XXX-X}
  \eisbn{978-3-96147-XXX-X}
  \issn{2625-9974}
  \printinformation{ggf. Satz \\
    ggf. Druck}
  \oralexam{12.12.1769}
  \dean{Daniel de Superville}
  \reviewer{Friedrich III. von
    Brandenburg-Bayreuth \\
    Christian Friedrich Carl Alexander von
    Brandenburg-Ansbach}
\end{codebox}

\faupressprintbibliography


\end{document}

%%% Local Variables: 
%%% mode: latex 
%%% TeX-engine: luatex
%%% TeX-master: t 
%%% End:

